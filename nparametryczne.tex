\section{\scshape M. Nieparametryczne}

\subsection{Wstęp}
\begin{frame}{Wady i zalety metod parametrycznych}
	\note{W idealnym dla naukowca świecie badane przez niego wartości podlegają dokładnie jednemu z typowych rozkładów matematycznych, który jest z góry znany. Niestety w praktyce rozkład trzeba zgadywać, a dane z reguły i tak nie będą do niego idealnie pasować. Rzeczywiste rozkłady są z reguły dużo bardziej złożone, ponieważ składa się na nie wiele czynników, a rozkłady teoretyczne pozwalają uwzględnić tylko część z nich, którą uznamy za najważniejszą. Takie przybliżenie może być wystarczająco dokładne, ale nie musi. Z drugiej strony jeżeli uda nam się dopasować rozkład, to otrzymujemy dużo większą moc, to znaczy potrzebujemy mniejszej próby do wyciągania wniosków z taką samą pewnością}
	Wady:
	\begin{itemize}
		\item Wymagają znajomości lub zgadywania rozkładu
		\item Rzadko pasują idealnie do danych
		\item Mała odporność na odstające wartości
	\end{itemize}
	Zalety:
	\begin{itemize}
		\item Posiadają większą moc statystyczną
	\end{itemize}
\end{frame}

\begin{Frame}{Metody nieparametryczne}
	\begin{itemize}
	\end{itemize}
\end{frame}