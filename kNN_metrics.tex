\section{\scshape kNN - metryki}

\subsection{Własno"sci}
\begin{frame}{Własno"sci metryki ($D$)}
\begin{itemize}
	\item nieujemno"sć: $D(a,b) \ge 0$
	\item zwrotno"sć: $D(a,b) = 0 \Leftrightarrow a = b$
	\item symetria: $D(a,b) = D(b,a)$
	\item $D(a,b) + D(b,c) \ge D(a,c)$
\end{itemize}
\end{frame}
\note{Metryka definuje sposób okre"slania odległo"sci od sąsiada. \\
Łatwo odnie"sć powyższe własno"sci do dystansu euklidesowego. \\
Nie może on być ujemny, odległo"sć jest równa $0$, gdy punkty są tożsame, a dystans między trzeba punktami musi zachowywać nierówno"sć trójkąta.}

\subsection{Rodzaje metryk}
\begin{frame}{Metryka euklidesowa}
\begin{equation}
	\centering
	(a,b) = \sqrt{\sum_{i=1}^n{a_i - b_i}^2}
\end{equation}
\begin{itemize}
	\item Intuicyjna metryka (dystans między punktami).
	\item Wrażliwa na transformacje przestrzeni.
\end{itemize}
\end{frame}

\note{Zwykła (powszechnie rozumiana) odległo"sć między punktami. \\
Po transformacji przestrzeni poprzez pomnożenie każdej współrzędnej, odległo"sci obliczane za pomocą wspomnianej metryki mogą bardzo odbiegać od pierwotnych. Wpływa to na sposób klasyfikacji. Stąd mała wrażliwo"sć metryki euklidesowej na transformacje przestrzeni.}

\begin{frame}{Metryka Manhattan}
\begin{equation}
	\centering
	D(a,b) = \sum_{i=1}^n{|a_i - b_i|}
\end{equation}
\begin{itemize}
	\item Suma projekcji odcinków łączacych punkty na osie układu.
	\item Wrażliwa na rotację, ale nie translację.
\end{itemize}
\end{frame}

\note{Nazywana \emph{Taxicab geometry} \\
Odległo"sć między punktami jest sumą warto"sci bezwzględnych różnic współrzędnych. Inaczej mówiąc - sumą projekcji odcinków łączących punkty na osie układu. \\
Nazwa \emph{taxicab} oraz \emph{Manhattan} pochodzi stąd, że wizualizacja dystansów przpomina rozkład nowojorskiej dzielnicy. Krzywe reprezentujące dystans są połączeniem odcinków leżących na bokach kwadratów takiej samej wielko"sci.}

\begin{frame}{Metryka Hamminga}
\begin{itemize}
	\item Dwa ciągi znaków równej długo"sci.
	\item Liczba pozycji, na których znaki są różne
	\item {\color[rgb]{1,0,0} 5}{\color[rgb]{0,1,0} 	  23}{\color[rgb]{1,0,0} 4}{\color[rgb]{0,1,0} 1} \\
    {\color[rgb]{1,0,0} 4}{\color[rgb]{0,1,0} 23}{\color[rgb]{1,0,0} 5}{\color[rgb]{0,1,0} 1} \\
		  dystans = 2
\end{itemize}
\end{frame}

\note{Reprezentuje minimalną liczbę zmian jakie należy wprowadzić do jednego z ciągów znaków, by był równy drugiemu. \\
Spełnia wszystkie niezbędne własno"sci. \\
Metryka Hamminga jest zdefiniowana dla ciągów znaków. Odpowiada metryce \emph{Manhattan} (dla wektorów).}

\begin{frame}{Metryka Tanimoto}
\begin{equation}
	\centering
	D(S_1,S_2) = \frac{n_1 + n_2 - 2n_12}{n_1 + n_2 - n_12}
\end{equation}
\begin{itemize}
	\item Użyteczna w taksonomii.
	\item Najlepiej działa dla takich samych lub zupełnie różnych zbiorów (bez gradacji).
\end{itemize}
\end{frame}

\note{$n_1$ i $n_2$ reprezentują liczbę elementów w zbiorach (odpowiednio) $S_1$ i $S_2$. $n_{12}$ to liczba elementów w obu zbiorach. \\
Metrykę wykorzystuje się w przypadkach, gdy nie trzeba okre"slać stopnia podobieństwa. Sprawdza się dla takich samych lub całkiem różnych zbiorów.
}